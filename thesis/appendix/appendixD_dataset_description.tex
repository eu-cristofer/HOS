% =============================================================================
% APPENDIX D: DATASET DESCRIPTION
% =============================================================================

\chapter{Dataset Description}

\section{Overview}

This appendix provides detailed descriptions of all datasets used in this research, including their characteristics, acquisition methods, and preprocessing procedures.

\section{Simulated Datasets}

\subsection{Signal Generation Parameters}

The simulated datasets were generated using the following parameters:

\begin{table}[H]
\centering
\caption{Simulation parameters for different fault types}
\label{tab:simulation_params}
\begin{tabular}{@{}lcccc@{}}
\toprule
Parameter & Normal & Unbalance & Bearing & Misalignment \\
\midrule
Sampling Frequency (Hz) & 12800 & 12800 & 12800 & 12800 \\
Signal Length (samples) & 8192 & 8192 & 8192 & 8192 \\
Rotational Speed (RPM) & 1800 & 1800 & 1800 & 1800 \\
Noise Level (SNR dB) & 20 & 20 & 20 & 20 \\
Number of Samples & 1000 & 1000 & 1000 & 1000 \\
\bottomrule
\end{tabular}
\end{table}

\subsection{Fault Simulation Models}

\subsubsection{Normal Operation}

The normal operation signal is modeled as:
\begin{equation}
x(t) = A_1 \sin(2\pi f_1 t + \phi_1) + n(t)
\end{equation}

where:
\begin{itemize}
    \item $A_1 = 1.0$ (amplitude)
    \item $f_1 = 30$ Hz (rotational frequency)
    \item $\phi_1 = 0$ (phase)
    \item $n(t)$ is Gaussian white noise
\end{itemize}

\subsubsection{Rotor Unbalance}

The unbalance signal includes harmonics:
\begin{equation}
x(t) = \sum_{k=1}^{5} A_k \sin(2\pi k f_1 t + \phi_k) + n(t)
\end{equation}

where the amplitudes $A_k$ are: [1.0, 0.3, 0.15, 0.08, 0.04]

\subsubsection{Bearing Defect}

The bearing defect signal includes characteristic frequencies:
\begin{equation}
x(t) = A_1 \sin(2\pi f_1 t) + A_{BPFO} \sin(2\pi f_{BPFO} t) + A_{BPFI} \sin(2\pi f_{BPFI} t) + n(t)
\end{equation}

where:
\begin{itemize}
    \item $f_{BPFO} = 107.4$ Hz (Ball Pass Frequency Outer)
    \item $f_{BPFI} = 162.2$ Hz (Ball Pass Frequency Inner)
    \item $A_{BPFO} = 0.5$, $A_{BPFI} = 0.3$
\end{itemize}

\subsubsection{Misalignment}

The misalignment signal includes 2X component:
\begin{equation}
x(t) = A_1 \sin(2\pi f_1 t) + A_2 \sin(2\pi \cdot 2f_1 t + \phi_2) + n(t)
\end{equation}

where $A_2 = 0.4$ and $\phi_2 = \pi/4$

\section{Real Datasets}

\subsection{Case Western Reserve University Bearing Data}

\subsubsection{Dataset Description}

The CWRU bearing dataset contains vibration data from a test rig with different bearing fault conditions.

\begin{table}[H]
\centering
\caption{CWRU dataset characteristics}
\label{tab:cwru_dataset}
\begin{tabular}{@{}ll@{}}
\toprule
Parameter & Value \\
\midrule
Sampling Frequency & 12,000 Hz \\
Signal Length & 120,000 samples (10 seconds) \\
Motor Speed & 1,797 RPM \\
Load & 0, 1, 2, 3 HP \\
Fault Diameter & 0.007, 0.014, 0.021 inches \\
Fault Location & Inner race, Outer race, Ball \\
\bottomrule
\end{tabular}
\end{table}

\subsubsection{Data Preprocessing}

The CWRU data was preprocessed as follows:
\begin{enumerate}
    \item Resampled to 12,800 Hz for consistency
    \item Segmented into 8,192 sample windows
    \item Applied detrending to remove DC offset
    \item Normalized to zero mean and unit variance
\end{enumerate}

\subsection{NASA Prognostics Data Repository}

\subsubsection{Dataset Description}

The NASA dataset contains bearing vibration data from accelerated life tests.

\begin{table}[H]
\centering
\caption{NASA dataset characteristics}
\label{tab:nasa_dataset}
\begin{tabular}{@{}ll@{}}
\toprule
Parameter & Value \\
\midrule
Sampling Frequency & 25,600 Hz \\
Test Duration & 35 days \\
Number of Bearings & 4 \\
Operating Conditions & Constant speed and load \\
Failure Modes & Inner race, Outer race, Ball \\
\bottomrule
\end{tabular}
\end{table}

\subsection{Industrial Dataset}

\subsubsection{Dataset Description}

Industrial data was collected from rotating machinery in a manufacturing facility.

\begin{table}[H]
\centering
\caption{Industrial dataset characteristics}
\label{tab:industrial_dataset}
\begin{tabular}{@{}ll@{}}
\toprule
Parameter & Value \\
\midrule
Sampling Frequency & 10,000 Hz \\
Machinery Type & Centrifugal pumps, Motors, Compressors \\
Operating Conditions & Variable speed and load \\
Fault Types & Unbalance, Misalignment, Bearing defects \\
Data Collection Period & 6 months \\
\bottomrule
\end{tabular}
\end{table}

\section{Data Augmentation}

\subsection{Augmentation Techniques}

To increase the dataset size and improve model generalization, several augmentation techniques were applied:

\begin{enumerate}
    \item \textbf{Noise Addition}: Gaussian white noise with different SNR levels
    \item \textbf{Time Shifting}: Random time shifts within the signal
    \item \textbf{Amplitude Scaling}: Random amplitude scaling factors
    \item \textbf{Frequency Modulation}: Slight frequency variations
\end{enumerate}

\subsection{Augmentation Parameters}

\begin{table}[H]
\centering
\caption{Data augmentation parameters}
\label{tab:augmentation_params}
\begin{tabular}{@{}lcc@{}}
\toprule
Technique & Parameter & Range \\
\midrule
Noise Addition & SNR (dB) & 15-25 \\
Time Shifting & Shift (samples) & ±100 \\
Amplitude Scaling & Scale Factor & 0.8-1.2 \\
Frequency Modulation & Frequency Shift (\%) & ±2 \\
\bottomrule
\end{tabular}
\end{table}

\section{Data Quality Assessment}

\subsection{Quality Metrics}

Data quality was assessed using several metrics:

\begin{table}[H]
\centering
\caption{Data quality metrics}
\label{tab:quality_metrics}
\begin{tabular}{@{}lccc@{}}
\toprule
Dataset & SNR (dB) & Completeness (\%) & Consistency Score \\
\midrule
Simulated & 20.0 & 100.0 & 1.00 \\
CWRU & 18.5 & 98.2 & 0.95 \\
NASA & 17.8 & 95.6 & 0.92 \\
Industrial & 16.2 & 89.3 & 0.88 \\
\bottomrule
\end{tabular}
\end{table}

\subsection{Data Validation}

Validation procedures included:
\begin{enumerate}
    \item Visual inspection of signal waveforms
    \item Statistical analysis of signal properties
    \item Frequency domain analysis
    \item Cross-validation with known fault conditions
\end{enumerate}

\section{Data Splitting Strategy}

\subsection{Training, Validation, and Test Sets}

The data was split as follows:

\begin{table}[H]
\centering
\caption{Data splitting strategy}
\label{tab:data_splitting}
\begin{tabular}{@{}lccc@{}}
\toprule
Dataset & Training (\%) & Validation (\%) & Test (\%) \\
\midrule
Simulated & 60 & 20 & 20 \\
CWRU & 60 & 20 & 20 \\
NASA & 60 & 20 & 20 \\
Industrial & 50 & 25 & 25 \\
\bottomrule
\end{tabular}
\end{table}

\subsection{Stratified Splitting}

Stratified splitting was used to ensure balanced representation of all fault classes in each subset.

\section{Data Preprocessing Pipeline}

\subsection{Preprocessing Steps}

The complete preprocessing pipeline includes:

\begin{enumerate}
    \item \textbf{Data Loading}: Load raw vibration signals
    \item \textbf{Quality Check}: Assess signal quality and completeness
    \item \textbf{Detrending}: Remove linear trends
    \item \textbf{Filtering}: Apply bandpass filter (10-1000 Hz)
    \item \textbf{Segmentation}: Divide into analysis windows
    \item \textbf{Normalization}: Normalize to zero mean and unit variance
    \item \textbf{Windowing}: Apply Hann window
    \item \textbf{Feature Extraction}: Extract features for analysis
\end{enumerate}

\subsection{Preprocessing Parameters}

\begin{table}[H]
\centering
\caption{Preprocessing parameters}
\label{tab:preprocessing_params}
\begin{tabular}{@{}ll@{}}
\toprule
Parameter & Value \\
\midrule
Bandpass Filter & 10-1000 Hz \\
Window Length & 8192 samples \\
Overlap & 50\% \\
Window Function & Hann \\
Normalization & Zero mean, unit variance \\
\bottomrule
\end{tabular}
\end{table}

\section{Data Storage and Access}

\subsection{File Organization}

The datasets are organized as follows:
\begin{itemize}
    \item \texttt{data/simulated/}: Simulated datasets
    \item \texttt{data/cwru/}: CWRU bearing data
    \item \texttt{data/nasa/}: NASA prognostic data
    \item \texttt{data/industrial/}: Industrial datasets
    \item \texttt{data/processed/}: Preprocessed datasets
\end{itemize}

\subsection{Data Formats}

Data is stored in multiple formats:
\begin{itemize}
    \item \textbf{CSV}: For feature matrices and metadata
    \item \textbf{NPZ}: For NumPy arrays (signals and features)
    \item \textbf{HDF5}: For large datasets
    \item \textbf{JSON}: For metadata and configuration
\end{itemize}

\section{Summary}

This appendix provides comprehensive documentation of all datasets used in this research. The datasets include both simulated and real-world data, covering various fault types and operating conditions. The preprocessing procedures ensure data quality and consistency across all datasets, enabling reliable evaluation of the proposed methodology.
