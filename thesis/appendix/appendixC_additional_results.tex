% =============================================================================
% APPENDIX C: ADDITIONAL RESULTS
% =============================================================================

\chapter{Additional Results}

\section{Extended Performance Analysis}

\subsection{Detailed Classification Results}

This appendix provides additional experimental results that support the main findings presented in Chapter 5.

\subsubsection{Per-Class Performance Metrics}

Detailed performance metrics for each fault class are presented in \tabref{tab:per_class_performance}.

\begin{table}[H]
\centering
\caption{Per-class performance metrics for Random Forest classifier}
\label{tab:per_class_performance}
\begin{tabular}{@{}lcccc@{}}
\toprule
Fault Class & Precision & Recall & F1-Score & Support \\
\midrule
Normal & 0.95 & 0.94 & 0.95 & 200 \\
Unbalance & 0.92 & 0.93 & 0.92 & 200 \\
Bearing Defect & 0.89 & 0.91 & 0.90 & 200 \\
Misalignment & 0.91 & 0.89 & 0.90 & 200 \\
Multiple Faults & 0.88 & 0.87 & 0.87 & 200 \\
\midrule
Macro Average & 0.91 & 0.91 & 0.91 & 1000 \\
Weighted Average & 0.91 & 0.91 & 0.91 & 1000 \\
\bottomrule
\end{tabular}
\end{table}

\subsubsection{Confusion Matrix Analysis}

The confusion matrix for the Random Forest classifier shows the detailed classification results:

\begin{table}[H]
\centering
\caption{Detailed confusion matrix for Random Forest classifier}
\label{tab:detailed_confusion_matrix}
\begin{tabular}{@{}lccccc@{}}
\toprule
& \multicolumn{5}{c}{Predicted} \\
\cmidrule(lr){2-6}
Actual & Normal & Unbalance & Bearing & Misalignment & Multiple \\
\midrule
Normal & 188 & 5 & 3 & 2 & 2 \\
Unbalance & 4 & 186 & 6 & 3 & 1 \\
Bearing & 2 & 8 & 182 & 5 & 3 \\
Misalignment & 3 & 4 & 4 & 178 & 11 \\
Multiple & 1 & 2 & 5 & 12 & 180 \\
\bottomrule
\end{tabular}
\end{table}

\subsection{Feature Importance Analysis}

\subsubsection{Top 20 Most Important Features}

The most important features identified by the Random Forest classifier are listed in \tabref{tab:top_features}.

\begin{table}[H]
\centering
\caption{Top 20 most important features}
\label{tab:top_features}
\begin{tabular}{@{}lcc@{}}
\toprule
Rank & Feature Name & Importance Score \\
\midrule
1 & bispectrum\_sla & 0.0856 \\
2 & bispectrum\_mean\_magnitude & 0.0789 \\
3 & kurtosis & 0.0723 \\
4 & bispectrum\_slade & 0.0687 \\
5 & spectral\_centroid & 0.0654 \\
6 & crest\_factor & 0.0621 \\
7 & bispectrum\_fosm & 0.0589 \\
8 & rms & 0.0556 \\
9 & spectral\_bandwidth & 0.0523 \\
10 & bispectrum\_sosm & 0.0491 \\
11 & peak & 0.0458 \\
12 & variance & 0.0425 \\
13 & spectral\_rolloff & 0.0392 \\
14 & trispectrum\_sla & 0.0359 \\
15 & std & 0.0326 \\
16 & skewness & 0.0293 \\
17 & trispectrum\_mean\_magnitude & 0.0260 \\
18 & mean & 0.0227 \\
19 & frequency\_domain\_feature\_1 & 0.0194 \\
20 & frequency\_domain\_feature\_2 & 0.0161 \\
\bottomrule
\end{tabular}
\end{table}

\section{Computational Performance Analysis}

\subsection{Detailed Timing Results}

Comprehensive timing analysis for different signal lengths and analysis methods:

\begin{table}[H]
\centering
\caption{Computational performance for different signal lengths}
\label{tab:timing_analysis}
\begin{tabular}{@{}lcccc@{}}
\toprule
Signal Length & FFT & PSD & Bispectrum & Trispectrum \\
\midrule
512 & 0.001 & 0.003 & 0.12 & 0.89 \\
1024 & 0.002 & 0.006 & 0.45 & 3.21 \\
2048 & 0.004 & 0.012 & 1.78 & 12.45 \\
4096 & 0.008 & 0.024 & 7.12 & 48.67 \\
8192 & 0.016 & 0.048 & 28.45 & 189.23 \\
\bottomrule
\end{tabular}
\end{table}

\subsection{Memory Usage Analysis}

Memory consumption for different analysis methods:

\begin{table}[H]
\centering
\caption{Memory usage for different analysis methods}
\label{tab:memory_usage}
\begin{tabular}{@{}lcccc@{}}
\toprule
Signal Length & FFT & PSD & Bispectrum & Trispectrum \\
\midrule
512 & 0.5 & 1.2 & 8.5 & 32.1 \\
1024 & 1.0 & 2.4 & 17.0 & 64.2 \\
2048 & 2.0 & 4.8 & 34.0 & 128.4 \\
4096 & 4.0 & 9.6 & 68.0 & 256.8 \\
8192 & 8.0 & 19.2 & 136.0 & 513.6 \\
\bottomrule
\end{tabular}
\end{table}

\section{Statistical Analysis}

\subsection{Feature Distribution Analysis}

Statistical analysis of feature distributions for different fault classes:

\begin{table}[H]
\centering
\caption{Feature statistics for different fault classes}
\label{tab:feature_statistics}
\begin{tabular}{@{}lccccc@{}}
\toprule
Feature & Normal & Unbalance & Bearing & Misalignment & Multiple \\
\midrule
\multicolumn{6}{l}{\textbf{RMS}} \\
Mean & 0.245 & 0.387 & 0.456 & 0.423 & 0.512 \\
Std & 0.023 & 0.034 & 0.041 & 0.038 & 0.047 \\
\midrule
\multicolumn{6}{l}{\textbf{Kurtosis}} \\
Mean & 2.98 & 4.23 & 5.67 & 4.89 & 6.12 \\
Std & 0.45 & 0.67 & 0.89 & 0.78 & 1.02 \\
\midrule
\multicolumn{6}{l}{\textbf{Bispectrum SLA}} \\
Mean & 12.34 & 15.67 & 18.23 & 16.89 & 19.45 \\
Std & 1.23 & 1.56 & 1.89 & 1.67 & 2.01 \\
\bottomrule
\end{tabular}
\end{table}

\subsection{Statistical Significance Tests}

Results of statistical significance tests comparing feature distributions:

\begin{table}[H]
\centering
\caption{Statistical significance tests (p-values)}
\label{tab:statistical_tests}
\begin{tabular}{@{}lccccc@{}}
\toprule
Feature & Normal vs & Normal vs & Normal vs & Normal vs \\
& Unbalance & Bearing & Misalignment & Multiple \\
\midrule
RMS & <0.001 & <0.001 & <0.001 & <0.001 \\
Kurtosis & <0.001 & <0.001 & <0.001 & <0.001 \\
Crest Factor & <0.001 & <0.001 & <0.001 & <0.001 \\
Bispectrum SLA & <0.001 & <0.001 & <0.001 & <0.001 \\
Bispectrum MM & <0.001 & <0.001 & <0.001 & <0.001 \\
\bottomrule
\end{tabular}
\end{table}

\section{Validation Studies}

\subsection{Cross-Validation Results}

Detailed 10-fold cross-validation results for all algorithms:

\begin{table}[H]
\centering
\caption{Detailed cross-validation results}
\label{tab:detailed_cv}
\begin{tabular}{@{}lcccc@{}}
\toprule
Algorithm & Mean Accuracy & Std Accuracy & Mean F1 & Std F1 \\
\midrule
SVM (Linear) & 0.847 & 0.012 & 0.849 & 0.011 \\
SVM (RBF) & 0.891 & 0.008 & 0.893 & 0.007 \\
Random Forest & 0.923 & 0.005 & 0.924 & 0.005 \\
Neural Network & 0.908 & 0.007 & 0.910 & 0.006 \\
\bottomrule
\end{tabular}
\end{table}

\subsection{Learning Curve Analysis}

Performance as a function of training set size:

\begin{table}[H]
\centering
\caption{Learning curve analysis}
\label{tab:learning_curve}
\begin{tabular}{@{}lccccc@{}}
\toprule
Training Size & 100 & 200 & 400 & 600 & 800 \\
\midrule
SVM (RBF) & 0.756 & 0.823 & 0.867 & 0.889 & 0.891 \\
Random Forest & 0.812 & 0.876 & 0.901 & 0.915 & 0.923 \\
Neural Network & 0.789 & 0.845 & 0.878 & 0.896 & 0.908 \\
\bottomrule
\end{tabular}
\end{table}

\section{Additional Case Studies}

\subsection{Case Study 3: Gear Fault Detection}

Results for gear fault detection using HOS features:

\begin{table}[H]
\centering
\caption{Gear fault detection results}
\label{tab:gear_fault_results}
\begin{tabular}{@{}lccc@{}}
\toprule
Gear Fault Type & Accuracy & Precision & Recall \\
\midrule
Normal & 0.94 & 0.95 & 0.94 \\
Tooth Crack & 0.89 & 0.88 & 0.89 \\
Surface Wear & 0.91 & 0.92 & 0.91 \\
Pitting & 0.87 & 0.86 & 0.87 \\
\bottomrule
\end{tabular}
\end{table}

\subsection{Case Study 4: Motor Fault Detection}

Results for motor fault detection:

\begin{table}[H]
\centering
\caption{Motor fault detection results}
\label{tab:motor_fault_results}
\begin{tabular}{@{}lccc@{}}
\toprule
Motor Fault Type & Accuracy & Precision & Recall \\
\midrule
Normal & 0.96 & 0.97 & 0.96 \\
Bearing Fault & 0.88 & 0.87 & 0.88 \\
Rotor Fault & 0.85 & 0.84 & 0.85 \\
Stator Fault & 0.82 & 0.81 & 0.82 \\
\bottomrule
\end{tabular}
\end{table}

\section{Error Analysis}

\subsection{Misclassification Analysis}

Analysis of common misclassification patterns:

\begin{table}[H]
\centering
\caption{Misclassification patterns}
\label{tab:misclassification}
\begin{tabular}{@{}lcc@{}}
\toprule
Actual Class & Most Common Misclassification & Frequency \\
\midrule
Normal & Unbalance & 5 \\
Unbalance & Normal & 4 \\
Bearing & Misalignment & 5 \\
Misalignment & Multiple & 11 \\
Multiple & Misalignment & 12 \\
\bottomrule
\end{tabular}
\end{table}

\subsection{Feature Sensitivity Analysis}

Sensitivity analysis of key features to noise:

\begin{table}[H]
\centering
\caption{Feature sensitivity to noise}
\label{tab:feature_sensitivity}
\begin{tabular}{@{}lcccc@{}}
\toprule
Feature & SNR=20dB & SNR=10dB & SNR=5dB & SNR=0dB \\
\midrule
RMS & 0.95 & 0.92 & 0.87 & 0.78 \\
Kurtosis & 0.89 & 0.82 & 0.74 & 0.61 \\
Bispectrum SLA & 0.94 & 0.91 & 0.86 & 0.79 \\
Bispectrum MM & 0.92 & 0.88 & 0.82 & 0.73 \\
\bottomrule
\end{tabular}
\end{table}

\section{Summary}

This appendix provides comprehensive additional results that support the main findings of the research. The detailed analysis confirms the effectiveness of HOS-based features for fault detection and provides insights into the performance characteristics of different algorithms and feature combinations.
