% =============================================================================
% APPENDIX A: MATHEMATICAL DERIVATIONS
% =============================================================================

\chapter{Mathematical Derivations}

\section{Derivation of Bispectrum Properties}

\subsection{Symmetry Properties}

The bispectrum $B_x(\omega_1, \omega_2)$ satisfies several symmetry properties that can be derived from the definition:

\begin{equation}
B_x(\omega_1, \omega_2) = \sum_{\tau_1=-\infty}^{\infty} \sum_{\tau_2=-\infty}^{\infty} C_{3,x}(\tau_1, \tau_2) e^{-j(\omega_1\tau_1 + \omega_2\tau_2)}
\end{equation}

\subsubsection{Conjugate Symmetry}

Starting from the definition of the third-order cumulant:
\begin{equation}
C_{3,x}(\tau_1, \tau_2) = E[x(t)x(t+\tau_1)x(t+\tau_2)]
\end{equation}

Taking the complex conjugate:
\begin{align}
C_{3,x}^*(\tau_1, \tau_2) &= E[x^*(t)x^*(t+\tau_1)x^*(t+\tau_2)] \\
&= E[x(t)x(t+\tau_1)x(t+\tau_2)] \quad \text{(for real signals)} \\
&= C_{3,x}(\tau_1, \tau_2)
\end{align}

Therefore:
\begin{align}
B_x^*(\omega_1, \omega_2) &= \sum_{\tau_1=-\infty}^{\infty} \sum_{\tau_2=-\infty}^{\infty} C_{3,x}^*(\tau_1, \tau_2) e^{j(\omega_1\tau_1 + \omega_2\tau_2)} \\
&= \sum_{\tau_1=-\infty}^{\infty} \sum_{\tau_2=-\infty}^{\infty} C_{3,x}(\tau_1, \tau_2) e^{j(\omega_1\tau_1 + \omega_2\tau_2)} \\
&= B_x(-\omega_1, -\omega_2)
\end{align}

\subsubsection{Interchange Symmetry}

From the symmetry of the cumulant:
\begin{equation}
C_{3,x}(\tau_1, \tau_2) = C_{3,x}(\tau_2, \tau_1)
\end{equation}

Therefore:
\begin{align}
B_x(\omega_1, \omega_2) &= \sum_{\tau_1=-\infty}^{\infty} \sum_{\tau_2=-\infty}^{\infty} C_{3,x}(\tau_1, \tau_2) e^{-j(\omega_1\tau_1 + \omega_2\tau_2)} \\
&= \sum_{\tau_1=-\infty}^{\infty} \sum_{\tau_2=-\infty}^{\infty} C_{3,x}(\tau_2, \tau_1) e^{-j(\omega_1\tau_1 + \omega_2\tau_2)} \\
&= \sum_{\tau_1=-\infty}^{\infty} \sum_{\tau_2=-\infty}^{\infty} C_{3,x}(\tau_1, \tau_2) e^{-j(\omega_2\tau_1 + \omega_1\tau_2)} \\
&= B_x(\omega_2, \omega_1)
\end{align}

\section{Derivation of HOS Estimation Variance}

\subsection{Variance of Bispectrum Estimate}

The variance of the bispectrum estimate using the direct method can be derived as follows.

For a signal $x[n]$ with $N$ samples, the bispectrum estimate is:
\begin{equation}
\hat{B}_x(\omega_1, \omega_2) = \frac{1}{N} \sum_{n=0}^{N-1} X(n)X(n+\omega_1)X^*(n+\omega_1+\omega_2)
\end{equation}

The variance is:
\begin{align}
\text{Var}[\hat{B}_x(\omega_1, \omega_2)] &= E[|\hat{B}_x(\omega_1, \omega_2)|^2] - |E[\hat{B}_x(\omega_1, \omega_2)]|^2
\end{align}

For Gaussian noise, the variance can be approximated as:
\begin{equation}
\text{Var}[\hat{B}_x(\omega_1, \omega_2)] \approx \frac{1}{N} P_x(\omega_1) P_x(\omega_2) P_x(\omega_1 + \omega_2)
\end{equation}

where $P_x(\omega)$ is the power spectral density.

\section{Derivation of Feature Extraction Formulas}

\subsection{Mean Magnitude of Bispectrum}

The mean magnitude of the bispectrum is defined as:
\begin{equation}
\text{MM} = \frac{1}{N^2} \sum_{\omega_1=0}^{N-1} \sum_{\omega_2=0}^{N-1} |B_x(\omega_1, \omega_2)|
\end{equation}

This can be derived from the definition of the bispectrum by taking the magnitude and averaging over all frequency pairs.

\subsection{Sum of Logarithmic Amplitudes}

The sum of logarithmic amplitudes is:
\begin{equation}
\text{SLA} = \sum_{\omega_1=0}^{N-1} \sum_{\omega_2=0}^{N-1} \log(|B_x(\omega_1, \omega_2)| + \epsilon)
\end{equation}

The small constant $\epsilon$ is added to avoid taking the logarithm of zero. This feature provides a measure of the overall "activity" in the bispectrum.

\subsection{Spectral Moments}

The first-order spectral moment is:
\begin{equation}
\text{FOSM} = \frac{\sum_{\omega_1=0}^{N-1} \sum_{\omega_2=0}^{N-1} \omega_1 |B_x(\omega_1, \omega_2)|}{\sum_{\omega_1=0}^{N-1} \sum_{\omega_2=0}^{N-1} |B_x(\omega_1, \omega_2)|}
\end{equation}

This represents the "center of mass" of the bispectrum in the $\omega_1$ direction.

\section{Derivation of Computational Complexity}

\subsection{Bispectrum Complexity}

The direct computation of the bispectrum requires:
\begin{itemize}
    \item Computing DFT: $O(N \log N)$
    \item Computing bispectrum: $O(N^2)$ for each frequency pair
    \item Total: $O(N^3)$
\end{itemize}

\subsection{Trispectrum Complexity}

Similarly, the trispectrum requires:
\begin{itemize}
    \item Computing DFT: $O(N \log N)$
    \item Computing trispectrum: $O(N^3)$ for each frequency triplet
    \item Total: $O(N^4)$
\end{itemize}

\section{Derivation of Rotor Response Equations}

\subsection{Unbalance Response}

For a rotor with unbalance, the equation of motion is:
\begin{equation}
M\ddot{q} + C\dot{q} + Kq = F_{unbalance}(t)
\end{equation}

The unbalance force is:
\begin{equation}
F_{unbalance}(t) = m \epsilon \omega^2 e^{j\omega t}
\end{equation}

where $m$ is the mass, $\epsilon$ is the eccentricity, and $\omega$ is the rotational frequency.

The steady-state response is:
\begin{equation}
q(t) = H(\omega) F_{unbalance}(\omega)
\end{equation}

where $H(\omega) = (K - \omega^2 M + j\omega C)^{-1}$ is the frequency response function.

\subsection{Critical Speed Analysis}

Critical speeds occur when the denominator of the frequency response function approaches zero:
\begin{equation}
\det(K - \omega^2 M + j\omega C) = 0
\end{equation}

This gives the natural frequencies of the system, which correspond to the critical speeds.
