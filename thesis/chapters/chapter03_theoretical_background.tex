% =============================================================================
% CHAPTER 3: THEORETICAL BACKGROUND
% =============================================================================

\chapter{Theoretical Background}

\section{Introduction}

This chapter establishes the mathematical foundations necessary for understanding and implementing higher-order spectral analysis in the context of rotordynamics fault detection. The theoretical framework covers signal processing fundamentals, higher-order statistics, spectral analysis techniques, and their specific applications to vibration signals from rotating machinery.

\section{Signal Processing Fundamentals}

\subsection{Continuous-Time Signals}

A continuous-time signal $x(t)$ is a function of a continuous variable $t$ representing time. For vibration analysis, we typically deal with real-valued signals representing displacement, velocity, or acceleration measurements.

\subsubsection{Signal Properties}

Important properties of signals include:

\begin{itemize}
    \item \textbf{Energy}: $E_x = \int_{-\infty}^{\infty} |x(t)|^2 dt$
    \item \textbf{Power}: $P_x = \lim_{T \to \infty} \frac{1}{2T} \int_{-T}^{T} |x(t)|^2 dt$
    \item \textbf{Autocorrelation}: $R_x(\tau) = \lim_{T \to \infty} \frac{1}{2T} \int_{-T}^{T} x(t)x(t+\tau) dt$
\end{itemize}

\subsection{Discrete-Time Signals}

In practice, continuous signals are sampled at discrete time intervals, resulting in discrete-time signals $x[n] = x(nT_s)$, where $T_s$ is the sampling period and $f_s = 1/T_s$ is the sampling frequency.

\subsubsection{Sampling Theorem}

The Nyquist-Shannon sampling theorem states that a signal can be perfectly reconstructed from its samples if the sampling frequency is at least twice the highest frequency component in the signal:

\begin{equation}
f_s \geq 2f_{max}
\end{equation}

\section{Higher-Order Statistics}

\subsection{Moments and Cumulants}

For a random process $x(t)$, the $k$-th order moment is defined as:

\begin{equation}
m_{k,x}(\tau_1, \tau_2, \ldots, \tau_{k-1}) = E[x(t)x(t+\tau_1) \cdots x(t+\tau_{k-1})]
\end{equation}

The $k$-th order cumulant is related to moments through the relationship:

\begin{align}
C_{1,x} &= m_{1,x} \\
C_{2,x}(\tau) &= m_{2,x}(\tau) - m_{1,x}^2 \\
C_{3,x}(\tau_1, \tau_2) &= m_{3,x}(\tau_1, \tau_2) - m_{1,x}[m_{2,x}(\tau_1) + m_{2,x}(\tau_2) + m_{2,x}(\tau_2-\tau_1)] + 2m_{1,x}^3
\end{align}

\subsection{Properties of Cumulants}

Cumulants possess several important properties:

\begin{enumerate}
    \item \textbf{Additivity}: For independent processes, cumulants are additive
    \item \textbf{Homogeneity}: $C_{k,ax}(\tau_1, \ldots, \tau_{k-1}) = a^k C_{k,x}(\tau_1, \ldots, \tau_{k-1})$
    \item \textbf{Symmetry}: Cumulants are symmetric functions of their arguments
    \item \textbf{Gaussian Suppression}: For Gaussian processes, cumulants of order $k > 2$ are zero
\end{enumerate}

\section{Higher-Order Spectral Analysis}

\subsection{Bispectrum}

The bispectrum is the two-dimensional Fourier transform of the third-order cumulant:

\begin{equation}
B_x(\omega_1, \omega_2) = \sum_{\tau_1=-\infty}^{\infty} \sum_{\tau_2=-\infty}^{\infty} C_{3,x}(\tau_1, \tau_2) e^{-j(\omega_1\tau_1 + \omega_2\tau_2)}
\end{equation}

\subsubsection{Properties of Bispectrum}

\begin{enumerate}
    \item \textbf{Symmetry}: $B_x(\omega_1, \omega_2) = B_x(\omega_2, \omega_1) = B_x^*(\omega_1, \omega_2)$
    \item \textbf{Periodicity}: $B_x(\omega_1, \omega_2) = B_x(\omega_1 + 2\pi, \omega_2) = B_x(\omega_1, \omega_2 + 2\pi)$
    \item \textbf{Quadratic Phase Coupling}: The bispectrum can detect quadratic phase coupling between frequency components
\end{enumerate}

\subsection{Trispectrum}

The trispectrum is the three-dimensional Fourier transform of the fourth-order cumulant:

\begin{equation}
T_x(\omega_1, \omega_2, \omega_3) = \sum_{\tau_1=-\infty}^{\infty} \sum_{\tau_2=-\infty}^{\infty} \sum_{\tau_3=-\infty}^{\infty} C_{4,x}(\tau_1, \tau_2, \tau_3) e^{-j(\omega_1\tau_1 + \omega_2\tau_2 + \omega_3\tau_3)}
\end{equation}

\subsection{Estimation of Higher-Order Spectra}

\subsubsection{Direct Method}

The direct method estimates the bispectrum as:

\begin{equation}
\hat{B}_x(\omega_1, \omega_2) = \frac{1}{N} \sum_{n=0}^{N-1} X(n)X(n+\omega_1)X^*(n+\omega_1+\omega_2)
\end{equation}

where $X(n)$ is the DFT of the signal $x[n]$.

\subsubsection{Indirect Method}

The indirect method first estimates the cumulant sequence and then computes its Fourier transform:

\begin{equation}
\hat{C}_{3,x}(\tau_1, \tau_2) = \frac{1}{N} \sum_{n=0}^{N-1} x[n]x[n+\tau_1]x[n+\tau_2]
\end{equation}

\section{Rotordynamics Theory}

\subsection{Rotor Dynamics Equations}

The equation of motion for a rotor system can be written as:

\begin{equation}
M\ddot{q} + C\dot{q} + Kq = F(t)
\end{equation}

where:
\begin{itemize}
    \item $M$ is the mass matrix
    \item $C$ is the damping matrix
    \item $K$ is the stiffness matrix
    \item $q$ is the displacement vector
    \item $F(t)$ is the force vector
\end{itemize}

\subsection{Unbalance Response}

For a rotor with unbalance, the response at frequency $\omega$ is:

\begin{equation}
X(\omega) = H(\omega)U(\omega)
\end{equation}

where $H(\omega)$ is the frequency response function and $U(\omega)$ is the unbalance force.

\subsection{Critical Speeds}

Critical speeds occur when the excitation frequency coincides with the natural frequencies of the system. At these speeds, the system response can become very large, potentially leading to failure.

\section{Fault Signatures in Vibration Signals}

\subsection{Rotor Unbalance}

Rotor unbalance typically produces:

\begin{itemize}
    \item Dominant frequency component at the rotational frequency
    \item Harmonic components at multiples of the rotational frequency
    \item Phase relationships between different measurement points
\end{itemize}

\subsection{Bearing Defects}

Bearing defects produce characteristic frequencies:

\begin{itemize}
    \item Ball Pass Frequency Outer (BPFO)
    \item Ball Pass Frequency Inner (BPFI)
    \item Ball Spin Frequency (BSF)
    \item Fundamental Train Frequency (FTF)
\end{itemize}

\subsection{Misalignment}

Misalignment typically produces:

\begin{itemize}
    \item High axial vibration
    \item 2X rotational frequency component
    \item Phase differences between horizontal and vertical measurements
\end{itemize}

\section{Feature Extraction from HOS}

\subsection{Bispectrum Features}

Common features extracted from the bispectrum include:

\begin{enumerate}
    \item \textbf{Mean Magnitude}: $\text{MM} = \frac{1}{N} \sum_{\omega_1, \omega_2} |B_x(\omega_1, \omega_2)|$
    \item \textbf{Sum of Logarithmic Amplitudes}: $\text{SLA} = \sum_{\omega_1, \omega_2} \log|B_x(\omega_1, \omega_2)|$
    \item \textbf{Sum of Logarithmic Amplitudes of Diagonal Elements}: $\text{SLADE} = \sum_{\omega} \log|B_x(\omega, \omega)|$
    \item \textbf{First-Order Spectral Moment}: $\text{FOSM} = \sum_{\omega_1, \omega_2} \omega_1 |B_x(\omega_1, \omega_2)|$
    \item \textbf{Second-Order Spectral Moment}: $\text{SOSM} = \sum_{\omega_1, \omega_2} \omega_1^2 |B_x(\omega_1, \omega_2)|$
\end{enumerate}

\subsection{Trispectrum Features}

Similar features can be extracted from the trispectrum:

\begin{enumerate}
    \item \textbf{Mean Magnitude}: $\text{MM} = \frac{1}{N} \sum_{\omega_1, \omega_2, \omega_3} |T_x(\omega_1, \omega_2, \omega_3)|$
    \item \textbf{Sum of Logarithmic Amplitudes}: $\text{SLA} = \sum_{\omega_1, \omega_2, \omega_3} \log|T_x(\omega_1, \omega_2, \omega_3)|$
\end{enumerate}

\section{Computational Considerations}

\subsection{Computational Complexity}

The computational complexity of HOS analysis is:

\begin{itemize}
    \item Bispectrum: $O(N^3)$ for direct computation
    \item Trispectrum: $O(N^4)$ for direct computation
\end{itemize}

where $N$ is the signal length.

\subsection{Efficient Algorithms}

Several approaches can reduce computational complexity:

\begin{enumerate}
    \item \textbf{Segmentation}: Divide long signals into shorter segments
    \item \textbf{Decimation}: Reduce sampling rate when appropriate
    \item \textbf{Parallel Processing}: Utilize multiple processors
    \item \textbf{FFT-based Methods}: Use FFT for efficient computation
\end{enumerate}

\section{Summary}

This chapter has established the theoretical foundations for higher-order spectral analysis in the context of rotordynamics fault detection. The mathematical framework provides the basis for implementing HOS-based fault detection algorithms, while the rotordynamics theory connects the signal processing concepts to the physical phenomena being analyzed.

The next chapter will present the methodology for applying these theoretical concepts to practical fault detection problems.
